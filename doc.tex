\begin{abstract}

\textbf{Abstract}: Null hypothesis significance testing remains popular despite decades of concern about misuse and misinterpretation.
We believe that a significant \mk{change word?}\jd{I did this on purpose (and also end with ``clarity''). Too cute? I kind of think it could help make the point.} part of the problem is due simply to language: significance testing has little to do with other meanings of the word ``significance''. 

Although null-hypothesis tests have limitations, we argue here that they remain useful in many contexts as a guide to whether a certain effect can be seen \emph{clearly} in that context (e.g.
whether we can clearly see that a correlation or between-group difference is positive or negative).
We therefore suggest that p-values resulting from null-hypothesis tests be described using the language of ``statistical clarity'' rather than ``statistical significance''.
We believe that this simple linguistic change has the potential to substantially enhance clarity in statistical communication.

\end{abstract}

\flushbottom
\maketitle
\newpage
\thispagestyle{empty}

\section*{Notes}

\mk{Not exactly what we are talking about here, but there are a few good ideas/quotes in this blog post: https://rapidecology.com/2018/05/02/abandon-anova-type-experiments/}

`` I’d like to propose that simple “comparison of means” experiments (those analyzed with a t-test or ANOVA) train our brains to think that this is the goal of science – to discover if an effect exists. --- But the goal of our science should be mechanistic and predictive models that make quantitative predictions that might be tested with experiments.'' 

``The absurdity of the t-test or  ANOVA way of doing science is apparent if something like temperature and CO2 are the experimental factors – for example in the many global climate change studies. What in ecology or physiology or cell biology is not related to temperature and CO2?''

\textbf{Structure:} I like the idea of two parallel boxes that have examples of language used in published papers on the
left and corrected language on the right.
Maybe three examples that correspond to four primary misinterpretations people
make (zero effect, large p-values as ``more'' evidence for something, difference between clear and not clear is not clear,
substantive significance)

\textbf{Previous Work:} The bottom line is that \emph{\textbf{a lot}} has been said on the general topic of NHST and 
p-values, though the idea of changing the language to clarity and retrieving the English definition of significant
is novel.
\citet{Bernardietal.2017} and \citet{McShaneetal.2017} have covered a lot of the history that is
recounted below.
It may be best to just cite these papers and state that concern about NHST and p-value abuse 
remains high.
\citep{Bernardietal.2017} present an analysis of the presence of errors in the interpretation of 
results and ambiguity in language focused on NHST. Out of  They point out that these results are similar to the
results from a similar review in the field of economics \citep{ZiliakandMcCloskey2008}.

\textbf{Signs}.
Who has said before that we test zero \emph{because} we want to know if we can see a sign clearly? If nobody, how much time do we want to spend on that point? 
From \citet{robinson2001past}, which arose from \citet{jones2000sensible}: [instead of no difference, or no statistical difference] "authors should instead simply state that `the direction of the differences among the treatments was undetermined' or that `the sign of the correlation among the two variables was undetermined.'" --- That is, "If p is greater than 0.05, the conclusion is simply that the sign of the difference is not yet determined"

\textbf{Notes:} A lot of this language assumes $H_{0} = 0$. Probably need to address clearly that many of these cited
papers are based in psychology.
I imagine we will want to anonymize quotes? (Though I kind of want to call out the null
result short courses paper --- maybe as a point about non-significance and calling for reform)

\textbf{Venue:} Forum in Trends in E and E: max 1200 words?
Note in The American Naturalist: max 3000 words? Perspectives article in Nature?
Rejected by Oikos forum!

\clearpage

\section*{Introduction}

For nearly a century statisticians and scientists have described the shortcomings of Null Hypothesis Significance Testing (NHST) \citep[see][]{Cohen1994}.
Books and articles proposing a deemphasis of the p~value in favor of effect sizes and confidence intervals have been cited thousands of times \citep{Cohen1994, Goodman1999, Wilkinson1999, ZiliakandMcCloskey2008, WassersteinandLazar2016}.
Yet, both reliance on NHST and the misinterpretation of results obtained from NHST remain high.

In a 2017 meta-analysis of 356 studies \citet{Bernardietal.2017} found that 72\% of studies contained an ambiguous use of the term ``significant'', 
49\% interpreted non-significant effects as zero effects, and 44\% failed to report an effect size in a comprehensible 
way.
The misuse and misinterpretation of NHST is so pervasive that there has been a recent push for drastically reducing 
\citep{SzucsandIoannidis2017} or abandoning \citep{McShaneetal.2017} its use.
Others have called for a more
strenuous significance threshold (e.g.
$p < 0.005$: \citealt{Benjaminetal.2018}); however, this change is unlikely to reduce
the misuse and misinterpretation of p~values, or the publication bias imposed by any p~value threshold \citep{Ridleyetal.2007}.

Instead, we suggest that using the language of ``statistical clarity'' would help researchers escape the trap of accepting 
the null hypothesis, reduce multiple types of misinterpretation associated with accepting the null hypothesis, and allow for
the continued use of NHST as a method of inference (e.g.
see \citealt{Abelson1997} for arguments for NHST).
The use of ``significance'' to describe the results of hypothesis tests is deeply, and sometimes subtly, misleading, because it is at odds with other meanings of the word: the P value is not an accurate gauge of whether a result is large in magnitude, biologically important, or relevant.
``Clarity,'' on the other hand, is an apt term for what hypothesis tests evaluate: a strong argument in favor of NHST is that testing the null hypothesis (even though we often know that it's false, and can never prove that it's true) is a proxy for asking whether we can see a difference clearly. In many cases, this comes down simply to whether we can be confident of the \emph{sign} of a difference or a correlation coefficient \cite{robinson2001past}.

Using the language of clarity makes sloppy thinking and misleading communication more difficult. 
For example, we often see sentences like, ``X et al. showed that there is no significant effect of Y on Z'' with the implication that this effect can now be assumed to be absent (or unimportant).
In fact, the sentence is erroneous even before we get to the implication: X et al.'s significance test provides information about their study, not about the underlying processes \cite{CanWeCiteThis}.
A better formulation would be ``X et al. did not find a significant effect of Y on Z.''
This mistake seems harder to make using the language of clarity:  ``X et al. showed that the effect of Y on Z is unclear'' is awkward.
Similarly, ``We did not find a clear difference in response between the control and sham groups'' seems both more colloquially accurate and harder to transform into something misleading than ``We did not find a significant difference\ldots''.
\citet{Bernardietal.2017} argued that ``...sociological and social significance are sacrificed on the altar of statistical significance''.
Describing statistical tests in terms of clarity could allow ``significant'' to reclaim its common English definition, and reduce conflation between statistical results and substantive significance.

\section*{Statistical Clarity}

\mk{Language here needs a lot of work.
Both in terms of clarity (haha) (use of ``confidence'' may be a bit confusing) and 
because I am not sure that everything I am saying here is 100\% correct.
Also not sure if this is how this section should
start.
Finally, portions of this have already been said in the paragraphs above.}

A p~value is not a property of reality (e.g.  a biological phenomena), but is a property of a data set and the statistical test used.
A p~value must therefore be interpreted and reported as a measure that quantifies our confidence about what we can see clearly about reality given our data.
A $p < 0.05$ tells us that we are confident in in our ability to clearly see the sign
of our effect (at an $\alpha = 0.05$ cutoff).
While a direct illustration of this clarity is shown with the confidence 
interval associated with $p < 0.05$, which will not overlap 0 (\mk{In nearly all cases?}), a p~value is a useful tool to 
quantify this clarity.
A p~value does not tell us about the substantive (e.g.
biological) importance of our finding.
Because
the p~value is a property of the data, and a true zero,``null effect'' does not exist in the vast majority of scientific 
fields \citep{Meehl1990, Tukey1991, Cohen1994}, $p < 0.05$ can be obtained for a very small effect with a large enough 
sample size.
Yet, references to a ``null effect'' can still be found in high impact journals \citep[e.g.][]{Feldonetal.2017}. 
To emphasize that effects are not truly zero, descriptions of statistical results using the language of clarity should begin 
with reference to the effect.
For example, ``The difference between the control and treatment group was not statistically 
clear.'' In Table~1 we provide examples from the literature of language that misinterprets p~values and a sensible way of
rewriting these statements to reduce confusion.

Another common mistake is using a larger p~value as stronger evidence for accepting the null hypothesis.
A larger p value 
actually corresponds to less clarity and thus the possibility of an undetected larger effect (see Table~1 for an example).
In 
2006 Gelman and Stern wrote that the difference between statistically significant and statistically non-significant is not 
itself significant.
Rewritten using ``clarity'' this statement indicates that if a single result is not statistically 
clear, any comparison made between that unclear result and a clear result will not itself be a clear comparison.
An example
taken from the literature is in Table~1. 

\mk{My table formatting skills are pretty bad}

\setlength\tabcolsep{1cm}.
\begin{tabular}{p{7.0cm}p{7.0cm}}
Language from published articles & Rewritten using ``clarity'' \\
\hline\\

\multicolumn{2}{c}{$\mathbf{p > 0.05 \neq}$ \textbf{zero effect}} \\

Toxins accumulate after acute exposure but have no effect on behaviour
& Toxins accumulate after acute exposure but effects on behaviour are statistically unclear
\\

There was no effect of elevated carbon dioxide on reproductive behaviors 
& The effect of elevated carbon dioxide on reproductive behaviors was statistically unclear
\\

The finding that species richness showed no significant relationship with the area of available habitat is surprising because richness is usually strongly influenced by landscape context 
& Although species richness is usually strongly influenced by landscape context, in the present study this relationship was not statistically clear 
\\ \\

\multicolumn{2}{c}{\textbf{Comparing P~values}}
\\

\ldots\ differences between treatment and control groups were nonsignificant, with P values of at least 0.3, and most in the range $0.7 \leq P \leq 0.9$.
& \ldots\ differences between treatment and control groups were not statistically clear ($P > 0.05$)
\\ \\

\multicolumn{2}{c}{\textbf{The difference between ``clear'' and ``not clear'' is ``not clear''}}
\\

This correlation was significant in males ($\rho=0.35$, P \textless 0.05) but not females ($\rho=0.35$, NS). \ldots [Afterwards the assumption is that a difference between males and females has been demonstrated]
& Although males and females show the same correlation coefficient ($\rho=0.35$), the sign of the coefficient is statistically clear only in males  \ldots [This phrasing may suggest to the authors that confidence intervals are called for.] 
\\

...risk of low BMD [bone mineral density] remained greater in HCV-coinfected women versus women with HIV alone
(adjusted OR 2.99, 95\% CI 1.33–6.74), but no association was found between HCV coinfection and low BMD in men 
(adjusted OR 1.26, 95\% CI 0.75–2.10). ... While the authors do not directly compare men and women they do state: 
The precise mechanisms for the association between viral hepatitis and low BMD in HIV-infected women but not men 
remain unclear.
& 
..risk of low BMD [bone mineral density] remained greater in HCV-coinfected women versus women with HIV alone
(adjusted OR 2.99, 95\% CI 1.33–6.74), but the association between HCV coinfection and low BMD in men was not
statistically clear (adjusted OR 1.26, 95\% CI 0.75–2.10). ... Pursuing the biological importance of HIV on BMD in 
women and not men is premature given these statistics and requires further study.
\\
\end{tabular}
   \captionof{table}{Rewriting misinformed language in the published literature using our proposed updated language.}

\section*{Closing Comments}

\subsection*{Non-inferiority tests}

Following a shift in language, we suggest using a non-inferiority test to test for a small effect.
Consider the following 
example.
A hospital is considering switching from bulky, expensive surgical face masks to lighter weight, cheaper masks.
A
clinical trail that tests the effectiveness of both masks and finds $p > 0.05$ is not a success; this tells the hospital 
that the difference between the masks is not statistically clear! What the hospital actually want to know is whether the 
cheaper, lighter masks are \emph{only a little bit worse than the bulky, expensive masks}.
This requires a $H_{0}$ that 
defines the target difference in effectiveness of the two masks that the hospital is willing to accept to make the 
transition.
We note that this statistical test should be set up as a one-tailed test; the logical, \emph{a priori} 
assumption is that the bulky, expensive masks are more effective.
In this case, a $p < 0.05$ tells the hospital that the 
difference between the new masks and old masks is small and that it they can proceed with the switch.

\subsection*{Focal vs non-focal hypotheses}

Accepting the null is especially common in diagnostic tests.
For example, a Shapiro-Wilk test of normality that
returns $p > 0.05$ means that it is statistically unclear how the data departs from normality (the data is collected
from a complicated world and does depart from normality).
Without the goal of picking on any paper in specific, 
``...allowing us to accept the null hypothesis about the normal spot count distribution.'' \citep{Karulinetal.2015} 
is common language that should be avoided.

\subsection*{Previous suggestions}

Following the lead of \citet{Cohen1994} and others \citep{Goodman1999, ZiliakandMcCloskey2008, WassersteinandLazar2016},
we conclude with a plea for an increased emphasis on effect sizes and confidence intervals.
However, we suggest that the
use of ``statistical clarity'' forces the \emph{need} for confidence intervals and effect sizes by removing ambiguity
from statements associated with $p < 0.05$. The statement ``The difference between our control and treatment groups was
statistically clear ($p = 0.007$)'' is unambiguously incomplete; an effect size and confidence interval 
is required to complete the story.

We echo previous statements in favor of ``neglected factors'' (prior and related evidence, plausibility of mechanism, 
study design and data quality, real world benefits, novelty and other factors) \citep{McShaneetal.2017} and 
reporting of \emph{a priori} analysis of statistical power to avoid emphasis on implausibly large effects given low 
statistical power \citep[the "winners curse"][]{GelmanandCarlin2014, SzucsandIoannidis2017, Bernardietal.2017}. 
Additionally, we are in favor of writing a statistical journal that chronicles all steps in the analysis process 
\citep{Kassetal.2016}, and clearly defining the boundary between results presented as inferences based on \emph{a priori}
hypotheses and those from \emph{post hoc} data exploration.


\section*{Lists}

\begin{itemize}
  \item Topics we may want to bring up
	  \subitem Including a p~value helps to avoid type-S errors
      \subitem usefulness of p --- use in splines: non-ridiculous standard to determine if we can see a pattern
      \subitem difficulty in interpreting confidence intervals?
\end{itemize}

\begin{itemize}
  \item Examples
  \subitem Equating $p > 0.05$ with no difference: \citep{Ortegoetal.2007, Bukovinszkyetal.2017, Singhetal.2017, Sundinetal.2017}
  \subitem Lack of clarity differentiating between biological and statistical significance: {Simonetal.2017}
  \subitem Complete lack of effect sizes: \citep{Juriadoetal.2017}
  \subitem Setting out to accept the null: \citep{Karulinetal.2015}
  \subitem ``...establish the \emph{a priori} selection of an effect level that is considered unacceptably toxic:'' \citep{Dentonetal.2011}
  \subitem With the \emph{correlation coefficient} for males and females (but significance for only one sex---``substrate 
  choice correlated significantly with food choice in males'') the authors go on to describe biologically why the 
  correlation would be significant for males but not females.
\citep{MerilaitaandJormalainen1997}
\end{itemize}

\begin{itemize}
  \item To do
  	\subitem unsort bib for convenient lookup of types of cites
    \subitem links in bib to pdfs
\end{itemize}

\subsection*{Acknowledgements}

We thank members of the Dushoff and Bolker labs for helpful comments on the first draft of the manuscript.


