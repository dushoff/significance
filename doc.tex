
\flushbottom
\maketitle
%\newpage
%\thispagestyle{empty}

%\clearpage

\linenumbers

\subsection*{Running Title}

Statistical Clarity

\subsection*{Word Count}

2,220

\subsection*{Abstract}

\begin{enumerate}

\item Null hypothesis significance testing (NHST) remains popular despite decades of concern about misuse and misinterpretation. There are many recent suggestions for mitigating problems arising from NHST, including calls for abandoning NHST in favor of Bayesian or information-theoretic approaches. We believe that NHST will continue to be widely used, and can be most usefully interpreted as a guide to whether a certain effect can be seen \emph{clearly} in a particular context (e.g. whether we can clearly see that a correlation or between-group difference is positive or negative). 
\item We believe that much misinterpretation of NHST is due to language: significance testing has little to do with other meanings of the word ``significance''. We therefore suggest that researchers describe the conclusions of null-hypothesis tests in terms of statistical ``clarity'' rather than ``significance''. We illustrate our point by rewriting common misinterpretations of the meaning of statistical tests found in the literature using the language of ``clarity''.
\item The meaning of statistical tests become easier to interpret and explain when viewed through the lens of ``statistical clarity''.
\item Our suggestion is mild, but practical: this simple semantic change could enhance clarity in statistical communication.

\end{enumerate}

\subsection*{Key Words}

\noindent Statistical philosophy; Statistical clarity; Hypothesis testing; \pval

\clearpage
\doublespacing

\subsection*{Introduction}

\noindent Statisticians and scientists have bemoaned the shortcomings of null hypothesis significance testing (NHST) for nearly a century \citep{Cohen1994}. Books and articles proposing the de-emphasis or abandonment of the \pval have been cited thousands of times \citep{Cohen1994, Goodman1999, Wilkinson1999, ZiliakandMcCloskey2008, WassersteinandLazar2016}. These works plead for a focus on effect sizes and confidence intervals, and point out that null effects that truly have zero magnitude are unrealistic or impossible in most fields outside of the hard physical sciences \citep{Meehl1990, Tukey1991, Cohen1994}. Yet, \pvals without confidence intervals (or even effect sizes) and references to null effects still pervade the scientific literature at all levels up to and including articles in high-impact journals.

In a meta-analysis of 356 studies \citet{Bernardietal.2017} found that 72\% of studies contained an ambiguous use of the term ``significant'', 49\% interpreted non-significant effects as zero effects, and 44\% failed to report a comprehensible effect size. The misuse and misinterpretation of NHST is so frequent that there have been recent calls for drastically reducing \citep{SzucsandIoannidis2017} or abandoning \citep{McShaneetal.2017} its use. Other prescriptions have included the complete abandonment of frequentist statistics \citep{The2011}, or the use of a stricter significance threshold (e.g. $p < 0.005$: \cite{Benjaminetal.2018}); however, the former seems impractical, while the latter is unlikely to reduce the misuse and misinterpretation of \pvals, or the publication bias imposed by any \pval threshold \citep{Ridleyetal.2007}.

Here, we argue that NHST remains useful, and that pervasive misuse can be reduced through a linguistic change: using the language of statistical ``clarity'' instead of statistical ``significance''.

\subsection*{The null hypothesis is false}

\noindent In most biological studies, the null hypothesis is known \emph{a priori} to be false. Even in cases where the null hypothesis is sensible (e.g., particle physics, \cite{Staley2017}), NHST does not provide evidence that a difference is exactly zero. This being the case, it is worth asking how NHST has survived ``if it is as idiotic as \ldots long believed'' \citet[cited in \cite{Kramer2011}]{ZiliakandMcCloskey2008}.

The value of NHST is that asking whether we can reject the null hypothesis is a proxy for asking whether we see clearly \emph{how} our data differs from it. For example, in a t-test, we are nominally asking whether we can see a \emph{difference} between two means, but the scientific question is whether we are confident which of the two means is \emph{larger} (and whether the size of the difference is biologically interesting); similarly, tests for whether two values are correlated are a proxy for whether we are confident about the \emph{sign} of the correlation coefficient \citep{robinson2001past}. In other cases (e.g., a one-way ANOVA), it may not be simple to describe the difference we see, but NHST is still a reasonable, widely accepted way to evaluate whether an effect has been seen clearly.

The ``idiocy'', if any, comes in the interpretive step. A statistical fact (``we have \emph{seen} a difference between the groups'', which should immediately prompt the question ``what have you learned \emph{about} that difference?'') is interpreted as a scientific fact (``there \emph{is} a `significant' difference between the groups''), which is often seen as an end in itself: ``we showed that the groups differ''.

\subsection*{The \pval is a property of the study}

\noindent Researchers often write sentences like, ``X et al. showed that there is no significant effect of Y on Z'' with the implication that this effect can now be assumed to be absent (or unimportant). In fact, the sentence is erroneous even before we get to the implication: significance tests provide information about \emph{a data set} -- that is, about a study, not about the study system \citep{HoenigandHeisey2001}. Indeed, a very small effect can lead to $p < 0.05$, when data is abundant (or noise is small); or a very large one can lead to $p > 0.05$ when the sample is small or noisy. 

The statement ``X et al. showed that Y has a statistically significant effect on Z'' is similarly misleading. Frequentist statistics effectively assume that the effect is present (or at least, admit that it can't be disproven). The question is whether it is seen in a particular data set. The statement ``X et al. were able to see the effect of Y on Z'' is not only more accurate, but it appropriately implies that something is missing: \emph{What} effect did they see?

\subsection*{Statistical clarity}

\noindent The language of ``statistical clarity'' could help researchers escape various logical traps while interpreting the results of NHST, allowing for the continued use of NHST as a simple, robust method of evaluating whether a data signal is clear (see \cite{Abelson1997} for arguments for NHST). The use of ``significance'' to describe the results of hypothesis tests is deeply, and sometimes subtly, misleading, because it is at odds with other meanings of the word: the \pval is not an accurate gauge of whether a result is large in magnitude, biologically important, or relevant. ``Clarity,'' on the other hand, is an apt term for what NHST actually evaluates. \citet{jones2000sensible} and \citet{robinson2001past} suggest that researchers should report $p > 0.05$ using language such as ``the direction of the differences among the treatments was undetermined''. This is a step in the right direction. Replacing ``significance'' with ``clarity'' takes this idea further, and has the potential to improve statistical communication.

For example, the sentence ``X et al. showed that the effect of Y on Z is statistically unclear'', is noticeably awkward. It seems less like a statement about the study system, and suggests the more straightforward ``did not find a statistically clear effect.'' Similarly, ``We did not find a clear difference in response between the control and sham groups'' is both more colloquial and harder to transform into a misleading statement than ``We did not find a significant difference \ldots''. \citet{Bernardietal.2017} complained that ``\ldots sociological and social significance are sacrificed on the altar of statistical significance''. Describing statistical tests in terms of clarity would allow ``significant'' to reclaim its common English definition and reduce conflation between statistical results and substantive significance.

Descriptions of statistical results using the language of clarity should begin with reference to the effect. For example, ``The difference between the control and treatment group was not statistically clear.'' Table~\ref{quotetab} shows published examples of statements that misinterpret \pvals in three different ways and demonstrates how to rephrase them in the language of clarity. We have attempted to do this thoughtfully, and therefore the language on the right differs from the language on the left by more than a simple substitution of ``significance'' to ``clarity''. We do not claim that executing a search-and-replace operation will automatically improve statistical practice; rather, we think it can prompt rethinking and reinterpretation. We also hope that, by drawing attention to effects, the language of clarity will encourage more reporting of effect sizes and confidence intervals.

\subsection*{Caveats}

Changing from ``significance'' to ``clarity'' should help researchers improve their statistical practice, but of course it cannot solve all of our problems with statistics. \bmb{repetitive with \P 3 of Conclusion?}The idea of ``statistical clarity'' will work best if it remains linked to the principles of attaching clarity to the study, rather than the system, and of focusing on effects and confidence intervals. If widely adopted, ``statistical clarity'' could eventually come to be seen as an end in itself, the way that ``significance'' is now; similarly, ``statistical clarity'' could be confounded with other meanings of  ``clarity'' as ``significance'' is now. We hope not, but in any case we feel that the unthinking use of ``clarity'' would be (marginally) better than the current unthinking use of ``significance''. We have found that scientists' understanding of our proposed use of statistical clarity is enhanced by explicitly connecting clarity statements to statements about P values.


\subsection*{Conclusions}

\noindent We believe that NHST is useful as a simple, robust way to ask whether an effect can be seen clearly in a particular data set \citep{robinson2001past}, and that careful, clarity-based language can reduce misinterpretation and miscommunication.

We agree with \citet{Cohen1994} and others \citep{Goodman1999, ZiliakandMcCloskey2008, WassersteinandLazar2016}, that scientific communication and understanding will be improved by a shift away from \pvals to effect sizes and confidence intervals. The use of ``statistical clarity'' should reinforce the need for confidence intervals and effect sizes by making bald statements about \pvals more obviously insufficient. The statement ``The difference between our control and treatment groups was not statistically clear ($p = 0.30$)'' is noticeably incomplete; an effect size and confidence interval are required to complete the story.

Improving language will not by itself solve all of the known problems with current statistical practice. We echo previous statements in favor of ``neglected factors'' (prior and related evidence, plausibility of mechanisms, study design and data quality, real world benefits, novelty, etc.) \citep{McShaneetal.2017} and reporting of \emph{a priori} analysis of statistical power to avoid emphasis on implausibly large effects given low statistical power \citep[the ``winner's curse''][]{GelmanandCarlin2014, SzucsandIoannidis2017, Bernardietal.2017}. Additionally, we support the writing of statistical journals that chronicle all of the steps in the analytical process \citep{Kassetal.2016}, and clearly delineating the boundary between inferences based on \emph{a priori} hypotheses and discoveries from \emph{post hoc} data exploration. These procedures help to avoid the ``garden of forking paths'' by which cryptic multiple testing amplifies noise to make it look like a signal of biologically interesting processes \citep{gelman_statistical_2014}). 

Whether or not our recommendations are broadly adopted by authors, reviewers, and editors, they can be useful for individual researchers who want to help themselves think clearly about NHST results. We have found that rephrasing NHST statements that we encounter (in the literature, or in seminar presentations) in terms of clarity has already helped us with both interpretation and communication.

\subsection*{Acknowledgments}

\noindent We thank members of the Dushoff and Bolker labs for helpful comments on the first draft of the manuscript.

